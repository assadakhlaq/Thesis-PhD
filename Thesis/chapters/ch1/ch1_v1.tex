\label{Chapter1}
 \setstretch{1.5}

\section{Introduction}
%In numerous signal processing applications, such as speech, magnetic resonance imaging (MRI) and radar imaging devices such as synthetic aperture radar (SAR), a quantity of primary interest is the phase of a received signal. For example, in radar applications, the phase may provide information about the distance to a target. An inherent property of the phase is that only its principal component is observed, that is, the observed value of the phase is always in the range $[-\pi, \pi)$. In applications, this leads to ambiguities in the value of some parameter of interest (such as the distance to a target). The task of rectifying these ambiguities is called \emph{phase unwrapping}. In this thesis we consider the problem of range or distance estimation and pioneer a novel approach to phase unwrapping that is based on a fundamental task in algebraic number theory, called \emph{closest lattice point}. This new approach promises to be both computationally simple, and statistically more accurate and robust than the current state-of-the-art phase unwrapping algorithms.

Range (or distance) estimation is important in various engineering applications such as global positioning system (GPS)~\cite{Teunissen_GPS_LAMBDA_2006,Teunissen_GPS_1995}, electronic surveying~\cite{Jacobs_ambiguity_resolution_interferometery_1981, anderson1998surveying}, and ranging cameras~\cite{time_of_flight_cam_continuous_wave_2009,Arrigo_patent_2014}. Among various available methods for range estimation~\cite{Chitte_RSS_Estimation2009, HingCheung_RSSbasedRangeEstimation2012, XinrongLi_TOA_range_estimation2004, Lanzisera_TOA_range_estimation2011}, phase of arrival based methods provide the most accurate estimates of the range~\cite{Fauzia_POA_range_estimation2007, Povalac_POA_rangeestimation2011}. 
%However, this methods inherit the problem of \emph{phase ambiguity}. Phase ambiguity problem occurs due to an inherent property of the phase that the observed value of the phase is always in the range $[-\pi, \pi)$. This phase ambiguity problem occurs when the unknown range is larger than the wavelength of the signal. This problem is addressed by using multiple different frequencies at the transmitter and observing the phase at each. The range can then be measured within an interval of length equal to the lcm of the wavelengths~\cite{Xiaowei_Li_robust_CRT_2009, W.Wang_closed_form_crt_2010, Li_distance_est_wrapped_phase,Akhlaq_basis_construction_range_est_2015}.
%Range (or distance) estimation is an important component in modern technologies such as electronic surveying~\cite{Jacobs_ambiguity_resolution_interferometery_1981, anderson1998surveying}, global positioning~\cite{Teunissen_GPS_LAMBDA_2006,Teunissen_GPS_1995}, and ranging cameras~\cite{time_of_flight_cam_continuous_wave_2009,Arrigo_patent_2014}. Common methods of range estimation are based upon received signal strength~\cite{Chitte_RSS_Estimation2009, HingCheung_RSSbasedRangeEstimation2012}, time of flight (or time of arrival)~\cite{XinrongLi_TOA_range_estimation2004, Lanzisera_TOA_range_estimation2011}, and phase of arrival~\cite{Fauzia_POA_range_estimation2007, Povalac_POA_rangeestimation2011}. Phase of arrival has become the technique of choice in modern high precision surveying, global positioning, and ranging cameras~\cite{Thangarajah_PDOA_rangeestimation2012, RTK_Report2003, Grejner-Brzezinska_ambguity-resolution2007, Odijk-nteger-ambiguity-resolutionPPP, time_of_flight_cam_continuous_wave_2009,Arrigo_patent_2014} because it provides the most accurate range estimates in many applications. %This thesis focuses on the phase of arrival based range estimators.
A difficulty with phase of arrival is that only the principal component of the phase can be observed, that is, the observed value of the phase is always in the range $[-\pi, \pi)$.    This is sometimes referred to as the problem of \emph{phase ambiguity} and it is related to what has been called the \emph{notorious wrapping problem} in the circular statistics and meteorology literature~\cite{Fisher1993}.  The task of rectifying these ambiguities is called \emph{phase unwrapping}. %This limits the range that can be unambiguously estimated. One approach to address this problem is to utilise signals of multiple different wavelengths and observe the phase at each.  The range can then be measured within an interval of length equal to the least common multiple of the wavelengths.  

In this thesis, we consider the problem of estimating the distance, or range, between two locations by measuring the phase of a sinusoidal signal transmitted between the locations. This method is only capable of unambiguously measuring range within an interval of length equal to the wavelength of the signal. To address this problem signals of multiple different wavelengths can be transmitted.  The range can then be measured within an interval of length equal to the least common multiple of these wavelengths. Range estimators from such observations have been studied by numerous authors.  Techniques include the beat wavelength method of Towers~et~al.~\cite{Towers_frequency_selection_interferometry_2003,Towers:04_generalised_frequency_selection}, the method of excess fractions~\cite{Falaggis_excess_fractions_2011,Falaggis_excess_fractions_2012,Falaggis_excess_fractions_2013,Falaggis_algebraic_solution_2014}, and methods based on the Chinese Remainder Theorem (CRT)~\cite{Oystein_Ore_general_chinese_Remainder_1952, Oded_Chinese_remaindering_with_errors_2000, Xia_generalised_CRT_2005, Xia2007, XWLi2008, W.Wang_closed_form_crt_2010, YangBin_range_estimation_with_CRT_2014, Xiao_multistage_crt_2014}.  Least squares/maximum likelihood and maximum a posteriori (MAP) estimators of range have been studied by Teunissen~\cite{Teunissen_GPS_1995}, Hassibi and Boyd~\cite{Hassibi_GPS_1998}, and more recently by Li~et~al.~\cite{Li_distance_est_wrapped_phase} and Akhlaq~et~al.~\cite{Akhlaq_basis_construction_range_est_2015}.  

This thesis focuses on the least squares estimation of range. A key realisation is that the least squares estimator can be efficiently computed by solving a well known integer programming problem, that of computing a \emph{closest point} in a \emph{lattice}~\cite{Agrell2002}.  Teunissen~\cite{Teunissen_GPS_1995} appears to have been the first to have realised this connection. Efficient general purpose algorithms for computing a closest lattice point require a~\emph{basis} for the lattice.  Constructing a basis for the least squares estimator of range is non-trivial.  Based upon the work of Teunissen~\cite{Teunissen_GPS_1995}, and under some assumptions about the distribution of phase errors, Hassibi and Boyd~\cite{Hassibi_GPS_1998} constructed a basis for the MAP estimator.  Their construction does not apply for the least squares estimator.\footnote{The least squares estimator is also the maximum likelihood estimator under the assumptions made by Hassibi and Boyd~\cite{Hassibi_GPS_1998}.  The matrix $G$ in~\cite{Hassibi_GPS_1998} is rank deficient in the least squares and weighted least squares cases and so $G$ is not a valid lattice basis.  In particular, observe that the determinant of $G$~\cite[p.~2948]{Hassibi_GPS_1998} goes to zero as the a priori assumed variance $\sigma_x^2$ goes to infinity.}  This is problematic because the MAP estimator requires sufficiently accurate prior knowledge of the range, whereas the least squares estimator is accurate without this knowledge.  

An explicit basis construction for the least squares estimator was recently given by Li~et.~al.~\cite{Li_distance_est_wrapped_phase} under the assumption that the wavelengths can be scaled to pairwise relatively prime integers. This assumption is impractical because it forces to use only the wavelengths that can be scaled to pairwise relatively prime integers. %In this thesis, we remove the need for this assumption and give an explicit construction in the general case.  
This affects the accuracy of the range estimator because the accuracy of the range estimator depends upon the wavelengths. It is possible that the wavelengths that give the most accurate range estimates could not be scaled to pairwise relatively prime integers. The dependence of the accuracy of range estimates upon the measurement wavelengths leads to two important questions:
\begin{itemise}
\item{Whether it is possible to devise a general basis construction method for the least squares range estimator that is independent of mutually co-prime restriction on the scaled wavelengths?}
\item{Given a general basis construction method, whether it is possible to select wavelengths that maximise the accuracy of the least squares range estimator?}
\end{itemise}
The first question is addressed in this thesis by employing some important properties from lattice theory. Solution to the first question naturally leads to the second important question of selecting wavelengths.
%This naturally leads to the problem of selecting wavelengths that maximise accuracy. 
However, the relationship between measurement wavelengths and range estimation accuracy is nontrivial and this complicates wavelength selection procedures. The selection procedure is typically subject to practical constraints such minimum and maximum wavelength (i.e. bandwidth constraints) and constraints on the maximum identifiable range. Procedures for wavelength selection have been described for the beat wavelength method~\cite{Towers_frequency_selection_interferometry_2003} and for the method of excess fractions~\cite{Falaggis_excess_fractions_2012}.  Some of these methods are heuristic and require a non-negligible amount of experimentation.  However, procedures for selecting wavelengths for the CRT and least squares range estimators have not yet been developed. The problem of selecting wavelengths that maximise the estimation accuracy of the least squares range estimator is also addressed in this thesis.

\section{Thesis Outline and Contributions}
Estimation of range using the phase of arrival method has extensively been studied in the literature using the beat wavelength method~\cite{Towers_frequency_selection_interferometry_2003,Towers:04_generalised_frequency_selection}, the method of excess fractions~\cite{Falaggis_excess_fractions_2011,Falaggis_excess_fractions_2012,Falaggis_excess_fractions_2013,Falaggis_algebraic_solution_2014}, methods based on the Chinese Remainder Theorem (CRT)~\cite{Oystein_Ore_general_chinese_Remainder_1952, Oded_Chinese_remaindering_with_errors_2000, Xia_generalised_CRT_2005, Xia2007, XWLi2008, W.Wang_closed_form_crt_2010, YangBin_range_estimation_with_CRT_2014, Xiao_multistage_crt_2014}, and the least squares/maximum likelihood and maximum a posteriori (MAP) estimators~\cite{Teunissen_GPS_1995, Hassibi_GPS_1998, Li_distance_est_wrapped_phase}. Each of these methods have some limitations when applied to practical systems. This thesis aims to provide a practically applicable solution based on the least squares estimation of range. A brief overview of the major contributions of each chapter is presented in the following.

\subsection*{Chapter 2 \textemdash~Background Overview}
This chapter presents a brief overview of concepts and background literature required for the understanding of this thesis. We first introduce some important concepts from lattice theory. The main focus of this section is on the Voronoi cell, the nearest lattice point problem, dual lattices, and the properties of lattices generated by intersection with or projection onto a subspace. These will be the most useful concepts used for the basis construction, robustness bound, and the wavelength selection method for the least squares range estimator described in Chapters 3, 4 and 5. %Next, we introduce the range estimation problem from multiple phase observations and 
Next, we present an overview of existing phase of arrival based range estimation methods such as the beat wavelength method, the method of excess fractions, and the CRT method. We also provide a highlight of wavelengths selection methods available in the literature for phase of arrival based range estimators. 

\subsection*{Chapter 3 \textemdash~Basis Construction for the Least Squares Range Estimator}
In this chapter we first introduce the system model used for the phase of arrival based range estimator that will be used throughout the thesis. We then derive the least squares estimator of the range using multiple phase observations at multiple frequencies. We show that how the solution to the least squares range estimator can be obtained by solving a problem from computational number theory called the closest lattice point problem. Finding a closest point in a lattice requires a basis for the lattice. Constructing a basis is non-trivial and an explicit construction has only been given in the case that the wavelengths can be scaled to pairwise relatively prime integers. In this chapter we present an explicit construction of a basis without this assumption on the wavelengths. This is important because the accuracy of the range estimator depends upon the wavelengths. The solution for the least squares range estimator provided in this chapter leads to a natural question of how to select a set of wavelengths that maximise the accuracy of the least squares range estimator.
\newline
This chapter includes material published in the following journal paper:
\newline
\begin{itemize}
\item{Assad~Akhlaq, R.~G.~McKilliam, and R.~Subramanian, ``{Basis Construction for Range Estimation by Phase Unwrapping},'' \emph{IEEE Signal Processing Letters,}  Vol. 22, No. 11, November 2015.}
\end{itemize}

\subsection*{Chapter 4 \textemdash~Robustness of the Least Square Range Estimator}
The least squares range estimator makes an estimate of the so called integer \emph{wrapping variables}. These wrapping variables are related to the whole number of wavelengths that occur over the range. Accurate estimators of the wrapping variables are also expected to be the accurate estimators of the range. This chapter derives an upper bound on the phase measurement errors such that if all absolute phase measurement errors are less than this bound, then the least squares range estimator is guaranteed to correctly estimate the wrapping variables and hence the range. This bound is derived using a lattice theory property called the \emph{inradius}. It is noted that this bound depends upon the values of the wavelengths used for range estimation. This naturally leads to the question of whether it is possible to select wavelengths that maximise the inradius. This is an interesting and nontrivial problem that we address in Chapter 5.\newline
This chapter includes material published in the following conference paper:
\newline
\begin{itemize}
 \item{Assad~Akhlaq, R.~G.~McKilliam, and Andr\'e Pollok, ``{Robustness of the Least Squares Range Estimator},'' in \emph{Proc. Australian Communications Theory Workshop (AusCTW),}  Melbourne, Australia, Jan 2016.}
\end{itemize}

\subsection*{Chapter 5 \textemdash~Selecting Wavelengths for Least Squares Range Estimation }
Motivated from the results in Chapter 3 and Chapter 4, in this chapter, we develop an algorithm to automatically select wavelengths for use with the least square range estimator. In this chapter, a key realisation is made that discloses the nontrivial relationship between the measurement wavelengths and the range estimation accuracy by relating the measurement wavelengths to the determinant of the lattice. This observation is utilised to design an optimisation criterion that is connected with the mean square error of the least squares range estimator. For this purpose interesting properties of a particular class of \emph{lattices} are used. These properties lead to a simple and sufficiently accurate approximation for the mean square range error in terms of the wavelengths. The resulting constrained optimisation problem is solved by a depth first search. \newline
This chapter includes material submitted in the following journal paper:
\newline
\begin{itemize}
\item{Assad~Akhlaq, R.~G.~McKilliam, R.~Subramanian and Andr\'e Pollok, ``{Selecting Wavelengths for Least Squares Range Estimation},'' submitted to \emph{IEEE Trans. Signal Processing,}  Jan. 2016.}
\end{itemize}

\subsection*{Chapter 6 \textemdash~Conclusion and Future Work}
This chapter briefly summarises the thesis contributions. The chapter concludes with a discussion on improving the frequency/wavelength selection algorithm and provides future research directions related to the work presented in this thesis.

In essence, this thesis provides an important research contribution for practical realisation of an efficient least squares range estimator. The basis construction method for the least squares range estimator provided in~\Chap{Chapter3} removes the restriction that the scaled wavelengths must be mutually co-prime integers.~\Chap{Chapter4} provides an upper bound on phase measurement errors to guarantee the correct estimation of wrapping variables. The relaxation on the use of measurement wavelengths and identification of an upper bound on phase measurement errors in Chapters 3 and 4 pave the path towards the selection of an optimised set of wavelengths to increase the accuracy of the least squares range estimator. A simple depth first search algorithm is provided in~\Chap{Chapter5} to select wavelengths that minimises the mean square error of the least squares range estimator. \Chap{Chapter6} concludes this thesis and provides future research directions.
