\label{Chapter6}
 \setstretch{1.5}

\section{Conclusions}
This thesis has provided new insight to the least squares estimation of range using the phase of arrival method. An important realisation made in the existing literature is that the least squares estimator can be efficiently computed by solving a well known integer programming problem, that of computing a closest point in a lattice. General purpose algorithms require a basis for the lattice to compute a closest lattice point. For the least squares estimator an explicit basis construction method was recently provided. This basis construction method is only valid if the measurement wavelengths can be scaled to pair-wise relatively prime integers. This assumption on wavelengths is not practically suitable because using only the wavelengths that can be scaled to pair-wise relatively prime integers may greatly degrade the accuracy of the range estimator. In fact, the accuracy of all the range estimators depends upon the measurement wavelengths. However, the relationship between the measurement wavelengths and the accuracy of these range estimators is not trivial. This thesis not only removes the restriction on scaled wavelengths to be co-prime but also provides an algorithm to automatically select wavelengths that maximise the accuracy of the least squares range estimator.

%The dependence of estimation accuracy upon the measurement wavelengths leads to 
%\begin{itemise}
%\item{Whether it is possible to devise a basis construction method that is independent of mutually co-prime restriction on the scaled wavelengths?}
%\item{Given a basis construction method that is  independent of mutually co-prime restriction on the scaled wavelengths, whether it is possible to select wavelengths that can maximise the accuracy of the least squares range estimator?}
%\end{itemise}
%
%These two questions are mainly answered in this thesis.

\Chap{Chapter2} presents a brief overview of some introductory concepts from lattice theory. The main concepts related to the Voronoi cell, the nearest lattice point problem, dual lattices, and the properties of lattices generated by intersection with or projection onto a subspace are discussed. These concepts provide a solid foundation for the evaluation of the least squares range estimator in later chapters. This chapter also provides a highlight of other range estimation techniques based on the phase of arrival method. This includes the techniques such as the beat wavelength method, the method of excess fractions, and the CRT method. Wavelengths selection methods for the beat wavelength method and the method of excess fractions are also discussed for the completeness of the topic.

\Chap{Chapter3} laid the foundation for the realisation of an algorithm to automatically select an optimised set of wavelengths. This chapter first provides the system model that is used throughout the thesis. The problem of range estimation from phase observations at multiple frequencies is described and a least squares estimator of the range is derived. The chapter then describes key connection between the least squares estimator and the \emph{closest lattice point} problem. It is shown that how the least squares estimator can be solved by computing a closest point in a lattice. The main contribution of this chapter is to provide a basis construction method that can be used to compute a closest point in a lattice. The basis construction method provided in this chapter is independent of the restriction that the the scaled wavelengths should be mutually co-prime integers. Moreover, this method is very simple as compared to the existing basis construction method. 

\Chap{Chapter4} first provides an insight about the correctness of the wrapping variables. This chapter derives an upper bound on the phase measurement errors such that if all absolute phase measurement errors are less than this bound, then the least squares range estimator is guaranteed to correctly estimate the wrapping variables and hence the range. An interesting property of the lattice called \emph{inradius} is used to derive this bound. It is observed that this bound is dependent upon the values of the measurement wavelengths. This observation combined with the result from~\Chap{Chapter3} that a basis can be constructed for a general set of wavelengths lead to a natural question of whether it is possible to select wavelengths that maximise the inradius and hence the accuracy of the least squares range estimator.

Motivated from the results in the previous chapters~\Chap{Chapter5} addresses the problem of selecting the wavelengths for the least squares estimator to maximise its accuracy. For this purpose the nontrivial relationship between the measurement wavelengths and the Voronoi cell of the lattice is exploited.  An important observation is that the Voronoi cell of a lattice can be approximated by a sphere of volume equal to that of the Voronoi cell in the presence of normally distributed phase noise. The volume of the Voronoi cell is equal to the determinant of the basis matrix. Our basis construction method provided a simple expression to calculate the determinant of the Voronoi cell. It is this simple expression that disclosed the nontrivial relationship between the measurement wavelengths and the Voronoi cell of the lattice. For the purpose of wavelengths selection we consider two approximations. The first approximates the error in the case that the least squares estimator of the wrapping variables is correct. The second upper bounds the probability that the wrapping variables are correct. We formulated an optimisation criterion that aims to minimise the mean square error of the estimator. This optimisation criterion is used to develop a wavelength selection algorithm that provides a set of wavelengths that typically result in smaller mean square error when used with the least square estimator. It is observed from numerical results that this algorithm outputs a set of wavelengths that outperforms the existing wavelengths selection methods for the excess fractions range estimator and the beat wavelength range estimators.

\section{Future Work}
This thesis has not only provided a new insight to the least squares estimation of range using phase of arrival method but has also opened up some extensions for future work. Some of these are summarised below.

\subsection*{Range estimation of moving target}
\Chap{Chapter3} deals with the estimation of range between two fixed nodes. A natural extension of this work is to extend it to range estimation of moving target. It would be interesting to see that is it possible to use lattice theoretic approach to estimate the range for a moving target? In this case, the least squares estimator of the range is nonlinear and it must be linearised to apply similar methods to range estimation of moving target. It is expected that solution to this problem will enhance the tracking of moving targets.

\subsection*{Selecting wavelengths using inradius of a lattice}
A natural extension of the research work in~\Chap{Chapter5} is to explore methods to use the inradius of the lattice to select wavelengths that maximise the accuracy of the least squares range estimator. For this purpose it may be possible to use lower bound~\ref{lowerboundinradius} on the inradius. However, the relationship between the wavelengths and the inradius is nontrivial and is an attractive future research problem. Another interesting problem will be to develop algorithms to select wavelengths that maximise the range estimation accuracy of moving targets.

The closest lattice point problem can be used in many other signal processing applications. A few of these are listed below.
\subsection*{Phase unwrapping and Image Processing}
There are many digital image processing techniques that can be used to extract the phase distribution from images generated by applications such as magnetic resonance imaging (MRI), synthetic aperture sonar and synthetic aperture radar (SAR) etc. In these applications extracted phase is related to the physical properties of the object under consideration. These digital image processing techniques employ arctangent function to extract the phase of the signal. The arctangent function produces outputs that are wrapped onto the range $[-\pi, \pi)$. Thus an unwrapping step must be added to the phase retrieval process to retrieve the true phase. An interesting problem is to use lattice theoretic approach to unwrap the phase in these digital image processing applications.

\subsection*{Lattice based cryptography}
Lattice-based cryptography is a term used for lattice based asymmetric cryptographic primitives. Although lattice based cryptography has beed a topic of interest to researchers for several decades, recently a renewed interest has emerged in lattice based cryptography with regard to quantum computer. Lattice based cryptosystems have recently appeared that are more resistant to attacks by both classical and quantum computers. It has appeared that breaking the lattice based cryptography is equivalent to solving known hard problems on lattices. \emph{There are currently no known quantum algorithms for solving lattice problems that perform significantly better than the best known classical (i.e., non-quantum) algorithms}~\cite{Ludwig2011}. In spite of this fact, there is still need for more research to develop improved cryptosystems to support the widespread use of lattice based cryptography.


























