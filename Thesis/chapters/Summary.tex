 \label{Summary}
 \setstretch{1.5}
%In numerous signal processing applications, such as speech, magnetic resonance imaging (MRI) and radar imaging devices such as synthetic aperture radar (SAR), a quantity of primary interest is the phase of a received signal. For example, in radar applications, the phase may provide information about the distance to a target. An inherent property of the phase is that only its principal component is observed. In applications, this leads to ambiguities in the value of some parameter of interest (such as the distance to a target). The task of rectifying these ambiguities is called \emph{phase unwrapping}. In this thesis we consider the problem of range (or distance) estimation that is is an important component in modern technologies such as electronic surveying, global positioning, and ranging cameras.
%
%and pioneer a novel approach to phase unwrapping that is based on a fundamental task in algebraic number theory, called the \emph{closest lattice point} problem. This new approach promises to be both computationally simple, and statistically more accurate and robust than the current state-of-the-art phase unwrapping methods for range estimation.

%Range (or distance) estimation is an important component in modern technologies such as electronic surveying, global positioning, and ranging cameras. Common methods of range estimation are based upon received signal strength, time of flight (or time of arrival), and phase of arrival. Phase of arrival has become the technique of choice in modern high precision surveying, global positioning, and ranging cameras because it provides the most accurate range estimates in many applications. 
%
%%An inherent property of the phase is that only its principal component is observed. This is sometimes referred to as the problem of \emph{phase ambiguity}. The task of rectifying this phase ambiguity is called \emph{phase unwrapping}. This thesis focuses on the problem of range (or distance) estimation using phase of arrival method and pioneers a novel approach to phase unwrapping that is based on a fundamental task in algebraic number theory, that of finding a \emph{closest point} in a \emph{lattice}. This new approach promises to be both computationally simple, and statistically more accurate and robust than the current state-of-the-art phase unwrapping methods for range estimation.
% 
%We consider the problem of estimating the distance, or range, between two locations by measuring the phase of a sinusoidal signal transmitted between the locations. This method is only capable of unambiguously measuring range within an interval of length equal to the wavelength of the signal. To address this problem signals of multiple different wavelengths can be transmitted.  The range can then be measured within an interval of length equal to the least common multiple of these wavelengths. Traditional estimators developed for optical interferometry include the beat wavelength and excess fractions methods.  Most recently, estimators based on the Chinese remainder theorem (CRT) and least squares have appeared.
%
%In this thesis we focus on the least squares estimation of the range. Least squares estimation of the range requires solution of a problem from computational number theory called the \emph{closest lattice point} problem.  Algorithms to solve this problem require a \emph{basis} for this lattice.  Constructing a basis is non-trivial and an explicit construction has only been given in the case that the wavelengths can be scaled to pairwise relatively prime integers.  In this thesis we present an explicit construction of a basis without this assumption on the wavelengths.  This is important because the accuracy of the range estimator depends upon the wavelengths.  Our results suggest that significant improvement in accuracy of the least squares range estimator can be achieved by using wavelengths that cannot be scaled to pairwise relatively prime integers.
% 
%This basis construction method naturally leads to the problem of selecting wavelengths that maximise accuracy.  Procedures for selecting wavelengths for the beat wavelength and excess fractions methods have previously been described, but procedures for the CRT and least squares estimators are yet to be developed.  Our basis construction method for the least squares range estimator laid the foundation for the development of an algorithm to automatically select wavelengths. The algorithm minimises an optimisation criterion connected with the mean square error. Interesting properties of a particular class of \emph{lattices} simplify the criterion allowing minimisation by depth first search.  Numerical results indicate that wavelengths that minimise this criterion can result is considerably more accurate range estimates than wavelengths selected by ad hoc means. It is expected that these results would be of great interest to readers in the areas of electronic surveying, global positioning and ranging cameras. 

Range (or distance) estimation is an important component in modern technologies such as electronic surveying, global positioning, and ranging cameras. Common methods of range estimation are based upon received signal strength, time of flight (or time of arrival), and phase of arrival. Phase of arrival has become the technique of choice in modern high precision technologies because it provides the most accurate range estimates in many applications. We consider the problem of estimating the distance, or range, between two locations by phase measurements of sinusoidal signals transmitted between these locations. Traditional phase of arrival based estimators developed for optical interferometry include the beat wavelength and excess fractions methods. Most recently, estimators based on the Chinese remainder theorem (CRT) and least squares have appeared.

In this thesis we focus on the least squares estimation of the range. Least squares estimation of range requires solution of a problem from computational number theory called the \emph{closest lattice point} problem. Algorithms to solve this problem require a \emph{basis} for this lattice. Constructing a basis is non-trivial and an explicit construction has only been given in the case that the wavelengths can be scaled to pairwise relatively prime integers. In this thesis we present an explicit construction of a basis without this assumption on the wavelengths. This is important because the accuracy of the range estimator depends upon the wavelengths. Results show that significant improvement in accuracy of the least squares range estimator can be achieved by using wavelengths that cannot be scaled to pairwise relatively prime integers. 

All range estimators, either explicitly or implicitly, make an estimate of so called \emph{wrapping variables} related to the whole number of wavelengths that occur over the range. We discover an upper bound such that if all absolute phase measurement errors are less than this bound, then the least squares range estimator is guaranteed to correctly estimate the wrapping variables. This bound is derived using a lattice theory property called the \emph{inradius}. It is noted that this bound depends upon the values of the wavelengths used for range estimation.

These findings naturally lead to the problem of selecting wavelengths that maximise the accuracy of the least squares estimator. Procedures for selecting wavelengths for the beat wavelength and excess fractions methods have previously been described, but procedures for the CRT and least squares estimators are yet to be developed. Our basis construction method for the least squares range estimator laid the foundation for the development of an algorithm to automatically select wavelengths. The algorithm minimises an optimisation criterion connected with the mean square error. Interesting properties of a particular class of \emph{lattices} simplify the criterion allowing minimisation by depth first search. Numerical results indicate that wavelengths that minimise this criterion can result is considerably more accurate range estimates than wavelengths selected by ad hoc means. It is expected that these results would be of great interest to readers in the areas of electronic surveying, global positioning and ranging cameras.
